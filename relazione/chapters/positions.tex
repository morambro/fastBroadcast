\section{Position changing}
\label{sec:position_change}

To simulate the presence of multiple devices, we generated a list of coordinates on a straight line, and dinamically assigned the right position to each device\footnote{The implementation is explained in the next section}. In figure \ref{fig:positions} is shown an example of how positions are dinamically reassigned. Suppose we have four devices, let's called them A, B, C and D, and suppose also we have eight different positions. At the beginning, we have A, B, C and D assigned rispectively on positions 1, 2, 3 and 4. At a certain point device A sends an \emph{Alert message} to all other three devices. As soon as it sends the messages, device A moves to position 5. Other devices compute contantion window and wait for a random time. Let's assume now that device C's timeout occours, so it forwards the message: as required by the algorithm, devices B and D stops themselves from forwarding data too, and device B moves also from position 2 to position 6, reconsidering the message again after moving. The process continues until last position available is reached. 

\begin{figure}[htbp]
\centering
\includegraphics[width=3.5in]{imgs/Positions_1.pdf}
\caption{Devieces position changes}
\label{fig:positions}
\end{figure}
