\subsection{Technologies}

\subsubsection{Android application}
Our Android application uses WI-FI connectivity to other devices, especially using \direct standard. Android SDK's APIs from 14 enables the use of this technology to create small wireless networks, where one of the devices is used as an \emph{Access Point}. \direct specification is only available for purchase from WI-FI Alliance\textsuperscript{\texttrademark} website\cite{wifi_direct}.
Basically a network created with Android \direct APIs is called \emph{group}. A \emph{group} has an owner (the access point), and one or more other devices. In the network the only available IP address is group owner's. 
Via APIs provided by the framework, user's permission is required in order to connect to other devices using \direct.

\subsubsection{Desktop application}
The Desktop version relies on low-level system calls to obtain a file descriptor associated with a Level 2 (Data Link) raw socket. In order to obtain real p2p connectivity without the need of an infrastructure (either physical or logical, like in the Android app) the communication must be based on MAC Addresses and the network interface must be in Monitor mode (so that all packets are forwarded to the socket for processing).
