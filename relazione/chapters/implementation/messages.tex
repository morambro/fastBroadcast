\subsection{Exchanged Messages}

Devices send messages encoded as XML strings; the structure of a message is show below in the following listing:
\begin{verbatim}
<message type="..." recipient_id="...">
    <content_block>
        <key>...</key>
        </content>...</content>
    </content_block>
    <content_block>
        <key>...</key>
        </content>...</content>
    </content_block>
    ...
    <content_block>
        <key>...</key>
        </content>...</content>
    </content_block>
</message>
\end{verbatim}
The messages we used contain data espressed as pairs \ttt{<key,content>}. All contained data can be easly converted to Hash Maps objects.

In our implementation, there are four different kinds of messages:
	\begin{description}
		\item[\textbf{Ping message (0)}] \hfill \\
		Application Control Message used to send MAC address to the \textit{group owner}.
		\item[\textbf{Map message (1)}] \hfill \\
		Application Control Message used by the group owner to send the complete \ttt{<MAC address,IP address>} pairs map to all the connected devices.
		\item[\textbf{Alert message (2)}] \hfill \\
		Protocol specific Message used to send an alert to be forwarded to the other connected devices.
		\item[\textbf{Hello message (3)}] \hfill \\
		Fast Broadcast protocol Message corresponding with Fast Broadcast algorithm \textit{Hello message}.
	\end{description}
