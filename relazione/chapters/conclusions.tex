\newpage
\section{Conclusions}

	Our Android framework represents a usable tool to develop testbed applications on common devices (e.g. Android smartphones or tablets). Unfortunately, due to \direct and, more specifically, Android \direct Framework implementation limits, we were forced to adopt several workarounds to simulate a true ad-hoc network communication: we exploited Android \direct framework to create a small network between the various peers and made them communicate via TCP/UDP Sockets, which imposes a heavy network load on the peer which acts as a server. Moreover, the TCP protocol's overhead and the execution under the Java Virtual Machine, with reduced computational power provided by mobile devices, introduce a noticeable slow down factor in simulations execution. 
	%results in a difficult validation of the the application results, and adds a significant amount of complexity to the coordination of different devices.
	Devices involved in the simulation can only be added at configuration time, but this in fact is not a problem, since an initial coordination between them is necessary in order to assign initial positions and the number of positions to skip when a device have to change his virtual location. 
	% TODO : da sistemare!!!
	The power of this framework version is noticeable when a real simulation is needed: exploiting devices' GPS system, vehicular networks protocols can be tested simply carrying Android devices on real vehicles. 
	
	Our Desktop framework adopts a faster and more dynamic method to transmit data: since in general vehicular algorithms do not need a complete network in order to exchange messages (message exchange nature is tipically broadcast), a possible solution involves the use of Data Link level Raw Sockets to implement a light, connectionless protocol in order to achieve a real ad-hoc network behaviour.% closer to the theoretical. 
	This kind of Socket was implemented on Fedora operating system (theoretically should work under every Linux distribution, but we experienced inconsistencies with Ubuntu distributions and its derivatives, so we recommend the use of a Fedora operating system). We also tried to port the Raw Socket library on a Nexus 7, but unfortunately the lack of essential modules and the inability to use Monitor Mode (it is not supported in most mobile network chipsets) resulted in a failure. 
	
	Future works will focus on ...  

