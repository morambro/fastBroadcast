\section{Introduction}
% The very first letter is a 2 line initial drop letter followed
% by the rest of the first word in caps.
% 
% form to use if the first word consists of a single letter:
% \IEEEPARstart{A}{demo} file is ....
% 
% form to use if you need the single drop letter followed by
% normal text (unknown if ever used by IEEE):
% \IEEEPARstart{A}{}demo file is ....
% 
% Some journals put the first two words in caps:
% \IEEEPARstart{T}{his demo} file is ....
% 
% Here we have the typical use of a "T" for an initial drop letter
% and "HIS" in caps to complete the first word.
\IEEEPARstart{R}{esearch} on Vehicolar Networks makes large use of software simulations, this because it is unfeasible to build a vehicular network of real vehicles that can act as relays of a message that could travel long distances.
To reduce costs, a possible solution is using common and small devices like smartphones to perform real tests carring them on vehicles, or to run simulations taking advantage of the features they offers.
To develop a complete testbed application for mobile devices, different features must be provided:
\begin{itemize}
	\item Possibility to use GPS location or fake positions.
	\item Wireless connectivity.
\end{itemize}

This features are very easy to supply: the most of the smartphones in commerce has GPS and WI-FI devices installed, and their Operating Systems provide APIs to use them.
Our work is exentially composed by two parts:
\begin{itemize}
	\item A distributed \textit{Android application}, which involves the use of a finite number of devices each running the same application. It lets users to specify fake locations for different devices with random distances between them, in order to simulate an high number of vehicles, and changes each device's virtual position when needed, or to use real GPS position in case of real tests. Our testbed application uses Android WI-FI technologies for devices communication, expecially \direct standard developed by WI-FI Alliance\textsuperscript{\texttrademark}.
	\item A Java Desktop application for laptops running Fedora. Only fake positions can be used with it, and for what regards connectivity, C Raw Socket where implemented to provide a connection and state free message sending via WI-FI. 
\end{itemize}

Both of our two products have the same architecture, that will be introduced in the next sections. To test our work, we used Fast Broadcast algorithm, which will be briefly introduced in section \ref{sec:fast_broadcast}.

