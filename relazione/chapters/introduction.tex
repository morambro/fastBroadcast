\section{Introduction}
% The very first letter is a 2 line initial drop letter followed
% by the rest of the first word in caps.
% 
% form to use if the first word consists of a single letter:
% \IEEEPARstart{A}{demo} file is ....
% 
% form to use if you need the single drop letter followed by
% normal text (unknown if ever used by IEEE):
% \IEEEPARstart{A}{}demo file is ....
% 
% Some journals put the first two words in caps:
% \IEEEPARstart{T}{his demo} file is ....
% 
% Here we have the typical use of a "T" for an initial drop letter
% and "HIS" in caps to complete the first word.
\IEEEPARstart{R}{esearch} on Vehicular Networks makes large use of software simulations, mainly because it is unfeasible to build a real-life vehicular network for every test project.
In order to reduce costs, a possible approach is to use common devices, like smartphones or PCs, from whitin each vehicle to perform real tests whitout the need of dedicated hardware built-in the cars, or even better, take advantage of the features they offer and build a reusable framework to run simulations.
A complete testbed application for mobile devices requires different features:
\begin{itemize}
	\item A GPS Antenna, or enough computational power to simulate location updates.
	\item Wireless connectivity.
\end{itemize}

Nowdays this features are very easy to find: most of the smartphones in the market already have GPS and WI-FI built-in, and the Operating Systems provides the needed APIs.
Given this, our work is consists in two parts:
\begin{itemize}
	\item A distributed \textit{Android application}, which involves the use of a finite number of devices each running the same application. It provides users with the ability to specify a list of locations for each device in order to simulate an high number of vehicles (each position representing a vehicle) and thus an easy way to modify virtual positions as needed, or to possibility of using real GPS positioning in case of real-life tests. Our testbed application uses Android WI-FI technologies for devices communication, specifically \direct standard developed by WI-FI Alliance\textsuperscript{\texttrademark}.
	\item A Java Desktop application (based on linux kernel and Fedora operating system). It offers the same capabilities as the Android application (with the exclusion of GPS positioning, although it can be easily implemented with a dedicated device driver), but uses Raw Sockets to reduce system and protocol overheads to a minimum and obtain a much more realistic simulation result. Morover, it is possible to run multiple instances of the application on the same machine (tested with up to 4), thus increasing the number of simulated vehicles.
\end{itemize}

Both our applications share the same architecture, that will be introduced in the next sections. The obtained framework was used to implement the Fast Broadcast algorithm, which will be briefly introduced in section \ref{sec:fast_broadcast}.

